\chapter{Introduzione}\label{ch:introduzione}
\section{Scopo del progetto}\label{sec:scopo-del-progetto}
Tutte le aziende, in particolare quelle di una certa complessità, hanno l'impellente necessità di gestire e razionalizzare le risorse e i servizi che erogano ai propri dipendenti, con particolare attenzione a quei servizi che permettono di utilizzare determinati strumenti aziendali e di implementare le comunicazioni ufficiali tra dipendente e azienda. 
Parte del processo di digitalizzazione delle aziende prevede, tra le altre iniziative, la creazione e l'utilizzo di sistemi automatici, intelligenti e fruibili sia da comouter che da periferica mobile per l'accesso ai servizi aziendali.\newline

L'azienda presso la quale ho svolto il mio tirocinio è \acrfull{gs}, un'azienda di più di 100 persone, di cui molte impiegate in via temporanea presso clienti in Italia e all'estero, ha interesse nel digitalizzare questi processi rapidamente e nel realizzare sistemi che permettano di aiutare altre aziende a fare altrettanto.\newline

In particolare, \acrlong{gs} vuole implementare una soluzione che permetta di gestire i propri strumenti e servizi aziendali in modo tale da garantirne un accesso rapido e efficiente per i propri dipendenti. Il servizio sul quale ho lavorato durante il tirocinio è stato quello di tracciamento e feedback delle attività formative erogate dall'azienda.
Questo sistema si presenta come una applicazione web, composta da elementi client e server, e è sviluppata tramite l'utilizzo di tecnologie avanzate come \gls{.net} per il back-end e Angular per il front-end. Si prevede, inoltre, la possibilità di integrare servizi di autenticazione e invio di email ai dipendenti.\newline

% Le attività da svolgere durante la tesi sono: Comprensione analisi del servizio (preparata da \acrlong{gs}) Apprendimento rapido tecnologie back-end e front-end fondamentali Progettazione e implementazione dell'infrastruttura di integrazione catalogo servizi. Sviluppo Servizio di tracciamento delle attività formative. Validazione funzionalità realizzate e test risultati.

TODO at the end\newline

Questo documento è organizzato nei seguenti capitoli.

Il capitolo 2 descrive, il contesto di riferimento, in particolare si parla del progetto scelto (del perchè e dello stato dell'arte), delle tecnologie utilizzate (per il back-end e per il front-end), delle metodologie di sviluppo e dei requisiti per portare a termine il progetto.


Il capitolo 3 è dedicato alla descrizione del prototipo realizzato, si parla dell'architettura del sistema nel dettaglio, delle tecnologie utilizzate e delle scelte progettuali fatte. Inoltre, si parla della fase di sviluppo e di test del prototipo.

Infine, il capitolo 4, presenta conclusioni e sviluppi futuri.
