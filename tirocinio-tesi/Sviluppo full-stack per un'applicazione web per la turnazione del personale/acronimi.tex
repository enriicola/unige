\usepackage[acronym]{glossaries}
\usepackage{acronym}
% \usepackage{glossaries-extra}

\makeglossaries% to compile the glossary
% \makenoidxglossaries

\newacronym{html}{HTML}{HyperText Markup Language}
\newacronym{cors}{CORS}{Cross-origin resource sharing}
\newacronym{http}{HTTP}{HyperText Transfer Protocol}
\newacronym{json}{JSON}{JavaScript Object Notation}
\newacronym{rest}{REST}{Representational State Transfer}
\newacronym{api}{API}{Application Programming Interface}
\newacronym{dbms}{DBMS}{Database Management System}
\newacronym{sql}{SQL}{Structured Query Language}
\newacronym{db}{DB}{database}
\newacronym{ide}{IDE}{Integrated Development Environment}
\newacronym{ui}{UI}{User Interface}
\newacronym{ux}{UX}{User Experience}
\newacronym{css}{CSS}{Cascading Style Sheets}
\newacronym{js}{JS}{JavaScript}
\newacronym{ts}{TS}{TypeScript}
\newacronym{npm}{NPM}{Node Package Manager}
\newacronym{cli}{CLI}{Command Line Interface}
\newacronym{gui}{GUI}{Graphical User Interface}
\newacronym{asp}{ASP}{Active Server Pages}
\newacronym{asp.net}{ASP.NET}{Active Server Pages.NET}
\newacronym{nosql}{NoSQL}{Not Only SQL}
\newacronym{dal}{DAL}{Data Access Layer}
\newacronym{sw}{SW}{software}
\newacronym{hw}{HW}{Hardware}
\newacronym{dom}{DOM}{Document Object Model}
\newacronym{xml}{XML}{Extensible Markup Language}
\newacronym{mvc}{MVC}{Model-View-Controller}
\newacronym{aot}{AoT}{Ahead-of-Time}
\newacronym{rxjs}{RxJS}{Reactive Extensions for JavaScript}
\newacronym{seo}{SEO}{Search Engine Optimization}
\newacronym{primeng}{PrimeNG}{PrimeFaces Next Generation}
\newacronym{fcl}{FCL}{Framework Class Library}
\newacronym{clr}{CLR}{Common Language Runtime}
\newacronym{i/o}{I/O}{Input/Output}
\newacronym{gs}{GS}{Gruppo SIGLA}
\newacronym{orm}{ORM}{Object-Relational Mapping}
\newacronym{url}{URL}{Uniform Resource Locator}
\newacronym{nvm}{NVM}{Node Version Manager}

% new terms
\newglossaryentry{framework}
{
    name=framework,
    description={In informatica, un framework, in italiano struttura o quadro, è un'architettura logica di supporto su cui un software può essere progettato e realizzato, spesso facilitandone lo sviluppo da parte del programmatore.}
}
\newglossaryentry{angular}
{
    name=Angular,
    description={Angular è un framework open source per lo sviluppo di applicazioni web con licenza MIT, evoluzione di AngularJS.}
}
\newglossaryentry{TypeScript}
{
    name=TypeScript,
    description={TypeScript è un linguaggio di programmazione open source sviluppato da Microsoft. È un super-set di JavaScript, che aggiunge tipizzazione statica e oggetti basati sulle classi.}
}
\newglossaryentry{Swagger}
{
    name=Swagger,
    description={Swagger è un framework open source per la progettazione, la creazione, la documentazione e l'uso di servizi Web RESTful.}
}
\newglossaryentry{deploy}
{
    name=deploy,
    description={Termine che indica tutte le attività necessarie per rendere disponibile un'applicazione software per l'utilizzo da parte di utenti finali.}
}
\newglossaryentry{API}
{
    name=API,
    description={Un'interfaccia di programmazione delle applicazioni, in acronimo API (dall'inglese application programming interface), è un insieme di definizioni e protocolli con i quali vengono realizzati e integrati software applicativi.}
}
\newglossaryentry{Dependency Injection}
{
    name=Dependency Injection,
    description={Un pattern architetturale per la gestione delle dipendenze tra gli oggetti, in cui un oggetto riceve le dipendenze da un'entità esterna anziché crearle direttamente.}
}
\newglossaryentry{Ahead-of-Time}
{
    name=Ahead-of-Time,
    description={La compilazione di un linguagio di programmazione ad alto livello in uno di livello inferiore prima dell'esecuzione di un programma, di solito al momento della composizione dello stesso, per ridurre la quantità di lavoro da eseguire al momento dell'esecuzione.}
}
\newglossaryentry{Indicizzazione}
{
    name=Indicizzazione,
    description={Un processo che consiste nel creare una struttura dati, generalmente una tabella hash o un albero, che permetta di accedere rapidamente ai dati in un database.}
}
\newglossaryentry{jquery}
{
    name=jQuery,
    description={jQuery è una libreria JavaScript per applicazioni web. Nasce con l'obiettivo di semplificare la selezione, la manipolazione, la gestione degli eventi e l'animazione di elementi DOM in pagine HTML, nonché implementare funzionalità AJAX.}
}
\newglossaryentry{angularjs}
{
    name=AngularJS,
    description={Un framework JavaScript open-source, principalmente sviluppato da Google e dalla comunità di sviluppatori individuali, che si occupa di estendere le funzionalità dell'HTML, utilizzato per la realizzazione di applicazioni web.}
}
\newglossaryentry{open source}
{
    name=open source,
    description={Un software open source è un software di cui gli autori (più precisamente i detentori dei diritti) rendono pubblico il codice sorgente, favorendone il libero studio e permettendone il miglioramento da parte di altri programmatori indipendenti.}
}
\newglossaryentry{template-driven}
{
    name=template-driven,
    description={Un approccio per la creazione di applicazioni web che si basa su template HTML, che vengono compilati lato client.}
}
\newglossaryentry{template}
{
    name=template,
    description={
        In informatica, un template è un modello predefinito che può essere utilizzato per creare documenti o pagine web, uno scheletro/schema che può essere riempito con contenuti specifici.
        }
}
\newglossaryentry{SEO}
{
    name=SEO,
    description={Un processo che consiste nel creare una struttura dati, generalmente una tabella hash o un albero, che permetta di accedere rapidamente ai dati in un database.}
}
\newglossaryentry{superset}
{
    name=superset,
    description={Un insieme che contiene tutti gli elementi di un altro insieme, e può contenere anche altri elementi.}
}
\newglossaryentry{transcompilare}
{
    name=transcompilare,
    description={Compilare codice sorgente, traducendolo dal suo linguaggio di programmazione in un qualsiasi altro, o in una versione più vecchia dello stesso linguaggio, producendo codice sorgente tradotto nel linguaggio di destinaiione.}
}
\newglossaryentry{fortemente tipato}
{
    name=fortemente tipato,
    description={Un linguaggio di programmazione è fortemente tipato se non permette operazioni tra tipi diversi.}
}
\newglossaryentry{refactoring}
{
    name=refactoring,
    description={Nell'ingegneria del software, il refactoring è una tecnica strutturata per modificare la struttura interna di porzioni di codice senza modificarne il comportamento esterno.}
}
\newglossaryentry{cross-platform}
{
    name=cross-platform,
    description={Un'applicazione cross-platform è un'applicazione software che è implementata e funziona su più di un sistema operativo.}
}
\newglossaryentry{inferenza}
{
    name=inferenza,
    description={L'inferenza è un processo di deduzione logica che permette di ottenere nuove informazioni a partire da quelle già note.}
}
\newglossaryentry{orientato agli oggetti}
{
    name=orientato agli oggetti,
    description={Un paradigma di programmazione che si basa sulla rappresentazione di oggetti e sulle relazioni tra di essi.}
}
\newglossaryentry{.net}
{
    name=.NET,
    description={.NET è un framework di sviluppo software, sviluppato da Microsoft, che fornisce un'ampia gamma di strumenti e librerie per lo sviluppo di applicazioni software.}
}
\newglossaryentry{input}
{
    name=input,
    description={Un input è un'informazione che viene fornita a un sistema informatico, in genere da un utente.}
}
\newglossaryentry{output}
{
    name=output,
    description={Un output è un'informazione che viene fornita da un sistema informatico, in genere a un utente.}
}
\newglossaryentry{software}
{
    name=software,
    description={L'insieme delle procedure e delle istruzioni in un sistema di elaborazione dati; si identifica con un insieme di programmi (in contrapposizione a hardware).}
}
\newglossaryentry{hardware}
{
    name=hardware,
    description={L'insieme delle componenti fisiche di un sistema di elaborazione dati.}
}
\newglossaryentry{visual basic}
{
    name=Visual Basic,
    description={Un linguaggio di programmazione orientato agli oggetti e agli eventi, multi-paradigma, sviluppato da Microsoft.}
}
\newglossaryentry{csharp}
{
    name=C\#,
    description={Un linguaggio di programmazione multi-uso,orientato agli oggetti, sviluppato da Microsoft. Principalmente, C\# è tipato staticamente, fortemente tipato, dichiarativo, imperativo, funzionale, ha contesto lessicale e discipline di porogrammazione orientata ai componenti.}
}
\newglossaryentry{fsharp}
{
    name=F\#,
    description={F\# è un linguaggio di programmazione funzionale, fortemente tipato, multi-paradigma, sviluppato da Microsoft.}
}
\newglossaryentry{containerizzazione}
{
    name=containerizzazione,
    description={Dall'inglese `containerization', è una tecnica di virtualizzazione a livello di sistema operativo che consente di eseguire più applicazioni in modo isolato su un singolo host. Differisce dalla virtualizzazione tradizionale in quanto non esegue un'intera macchina virtuale, ma esclusivamente l'applicazione e le sue dipendenze.}
}
\newglossaryentry{IDE}
{
    name=IDE,
    description={Dall'inglese `Integrated Development Environment', è una tipologia di programma per sviluppatori che offre un insieme di strumenti per la scrittura, la compilazione e il debug del codice sorgente.}
}
\newglossaryentry{just-in-time}
{
    name=just-in-time,
    description={Un compilatore just-in-time (JIT) è un compilatore che trasforma il codice sorgente in codice macchina (binario) al momento preciso dell'esecuzione.}
}
\newglossaryentry{tupla}
{
    name=tupla,
    description={
        In matematica, una tupla è una sequenza ordinata di elementi. In informatica, una tupla è una struttura dati che rappresenta una collezione di elementi, ognuno dei quali può essere di un tipo diverso.}
}
\newglossaryentry{migrazione}
{
    name=migrazione,
    description={
        % Una migrazione è il processo di trasferimento di dati da un sistema a un altro.
        Le migrazioni sono un modo per aggiornare il database in modo incrementale, in modo che possa essere mantenuto e aggiornato senza doverlo ricreare ogni volta che cambia il modello di dati.
        }
}
\newglossaryentry{proprietà}
{
    name=proprietà,
    description={
        In C\#, una proprietà è un membro di una classe che fornisce un modo flessibile per leggere, scrivere o calcolare il valore di un campo privato.
        }
}
\newglossaryentry{getter}
{
    name=getter,
    description={
        Un getter è un metodo, incapsulato in una proprietà, che \textbf{legge} il valore di un campo privato di una classe.
        }
}
\newglossaryentry{setter}
{
    name=setter,
    description={
        Un setter è un metodo, incapsulato in una proprietà, che \textbf{scrive} il valore di un campo privato di una classe.
        }
}
\newglossaryentry{heap}
{
    name=heap,
    description={
        Lo heap è una regione di memoria in cui vengono allocati gli oggetti istanziati durante l'esecuzione di un programma.
        }
}
\newglossaryentry{backup}
{
    name=backup,
    description={
        Una copia di sicurezza dei dati, solitamente memorizzata in un luogo diverso da quello in cui si trovano i dati originali.
        }
}
\newglossaryentry{debug}
{
    name=debug,
    description={
        Il debug è il processo di individuazione e correzione degli errori in un programma.
        }
}
\newglossaryentry{query}
{
    name=query,
    description={
        Una richiesta di informazioni o di azioni da parte di un database; solitamente scritta in un linguaggio di interrogazione, come ad esempio \acrshort{sql}.
        }
}
\newglossaryentry{build}
{
    name=build,
    description={
        Il processo di compilazione di un programma, ovvero la trasformazione del codice sorgente in codice eseguibile, in questo riferita alla compilazione eterogenea di tutti i file del backend.
        }
}
\newglossaryentry{form}
{
    name=form,
    description={
        Un form è un'interfaccia grafica che permette all'utente di inserire dati in un'applicazione web
        }
}